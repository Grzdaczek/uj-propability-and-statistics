\documentclass[11pt]{extarticle}
\usepackage[a4paper, margin=1in]{geometry}
\usepackage{multicol}
\usepackage{float}
\usepackage{amsmath}
\usepackage{lipsum}
\usepackage{mathtools}
\usepackage{cuted}
\usepackage[T1]{fontenc}
\usepackage[polish]{babel}
\usepackage[utf8]{inputenc}
\author{Grzegorz Janysek}
\title{Obracanie dustrybuanty - Zadanie 3.3}

\begin{document}

	\maketitle

	Zadana funkcja gęstości prawdopodobieństwa

	\begin{align}
		f_X(x) = \begin{cases} 
			0 							& x \in (-\infty, -1] \\
			\frac{1}{3}x + \frac{1}{3} 	& x \in (-1, 0] \\
			\frac{1}{3} 				& x \in (0, 2] \\
			-\frac{1}{3}x + 1 			& x \in (2, 3] \\
			0 							& x \in (3, +\infty)
		\end{cases}
	\end{align}
	
	Obliczanie dustrybuanty \( F_X(x) \) na podstawie funkcji gęstości prawdopodobieństwa \( f_X(x) \)

	\begin{align}
		F_X(x) = \int_{-\infty}^x f_x(t)dt
	\end{align}

	\begin{align}
		&\text{dla } x \in (-\infty, -1] :&
			F_X(x) &= 0 \\
		&\text{dla } x \in (-1, 0] :&
			F_X(x) &=
			F_X(-1) +
			\int_{-1}^{x} (\frac{1}{3}t + \frac{1}{3}) dt =
			\frac{1}{6}x^2 + \frac{1}{3}x + \frac{1}{6} \\
		&\text{dla } x \in (0, 2] :&
			F_X(x) &=
			F_X(0) +
			\int_0^x \frac{1}{3}dt =
			\frac{1}{3}x + \frac{1}{6} \\
		&\text{dla } x \in (2, 3] :&
			F_X(x) &= F_X(2) + \int_2^x (-\frac{1}{3}t + 1) dt =
			-\frac{1}{6}x^2 + x - \frac{1}{2} \\
		&\text{dla } x \in (3, +\infty) :&
			F_X(x) &= 1
	\end{align}

	\begin{align}
		F_X(x) = \begin{cases} 
			0 												& x \in (-\infty, -1] \\
			\frac{1}{6}x^2 + \frac{1}{3}x + \frac{1}{6}		& x \in (-1, 0] \\
			\frac{1}{3}x + \frac{1}{6}						& x \in (0, 2] \\
			-\frac{1}{6}x^2 + x - \frac{1}{2}				& x \in (2, 3] \\
			1 												& x \in (3, +\infty)
		\end{cases}
	\end{align}

	\pagebreak

	Obliczanie funkcji odwrotnej do \( F_X(x)\) dla \( x \in (-1, 0] \implies F_X(x) \in (0, \frac{1}{6}) \)

	\begin{align}
		F_X(y) &= \frac{1}{6}y^2 + \frac{1}{3}y + \frac{1}{6} \\
		x &= \frac{1}{6}y^2 + \frac{1}{3}y + \frac{1}{6} \\
		x &= \frac{1}{6} (y^2 + 2y + 1) \\
		6x &= (y + 1)^2 \\
		\sqrt{6x} - 1 &= y
	\end{align}

	\begin{align}
		\text{stąd, dla } x \in (0, \frac{1}{6}) : F_X^{-1}(x) = \sqrt{6x} - 1
	\end{align}

	Postępowanie wygląda alanlogicznie dla pozostałych zakresów, otrzymujemy:

	\begin{align}
		f_X(x) = \begin{cases} 
			\sqrt{6x} - 1 		& x \in (0, \frac{1}{6}] \\
			3x - \frac{1}{2}	& x \in (\frac{1}{6}, \frac{5}{6}] \\
			3 - \sqrt{6-6x} 	& x \in (\frac{5}{6}, 1] \\
		\end{cases}
	\end{align}

	Przekształcenie zmennej losowej jednorodnej za pomocą \( F_X^{-1}(x) \)
	pozwala na uzyskanie zmiennej losowej o zadanej funkcji gęstości prawdopodobieństwa \( f_x(x) \)
	
\end{document}